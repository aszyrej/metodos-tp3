\documentclass[11pt, a4paper]{article}
%\usepackage[spanish]{babel}
\usepackage{a4wide}
\usepackage{amsfonts}
\usepackage{graphicx}
\usepackage{verbatim}
\usepackage{todonotes}
\usepackage{amsmath}
\usepackage{url}
\usepackage[spanish]{babel}

\usepackage{tikz}
\usetikzlibrary{decorations.markings,arrows}

\parindent = 0 pt
\parskip = 11 pt

\newcommand{\real}{\hbox{\bf R}}

\begin{document}
\begin{center}
\begin{tabular}{r|cr}
 \begin{tabular}{c}
{\large\bf\textsf{\ M\'etodos Num\'ericos\ }}\\ 
Segundo Cuatrimestre 2013\\
{\bf Trabajo Pr\'actico 3}
\end{tabular} &
\begin{tabular}{@{} p{1.6cm} @{}}
\includegraphics[width=1.6cm]{../../../logodpt.jpg}
\end{tabular} &
\begin{tabular}{l @{}}
 \emph{Departamento de Computaci\'on} \\
 \emph{Facultad de Ciencias Exactas y Naturales} \\
 \emph{Universidad de Buenos Aires} \\
\end{tabular} 
\end{tabular}
\end{center}

\vskip 25pt
\hrule
\vskip 11pt
 
\textbf{Introducci\'on}

A partir de la evoluci\'on de Internet durante la d\'ecada de 1990, el desarrollo de motores de b\'usqueda se ha convertido
en uno de los aspectos centrales para su efectiva utilizaci\'on. Hoy en d\'ia, sitios como Yahoo, Google y Bing ofrecen
distintas alternativas para realizar b\'usquedas complejas dentro de un red que contiene miles de millones de p\'aginas
web. 

En sus comienzos, una de las caracter\'isticas que distngui\'o a Google respecto de los motores de b\'usqueda de la \'epoca
fu\'e la calidad de los resultados obtenidos, mostrando al usuario p\'aginas relevantes a la
b\'usqueda realizada. El esquema general de los or\'igenes de este motor de b\'usqueda es brevemente explicando en 
Brin y Page \cite{Brin1998}, donde se mencionan aspectos t\'ecnicos que van desde la etapa de obtenci\'on de
informaci\'on de las p\'aginas disponibles en la red, su almacenamiento e indexado y su posterior procesamiento,
buscando ordenar cada p\'agina de acuerdo a su importancia relativa dentro de la red. El algoritmo utilizado para esta
\'ultima etapa es denominado PageRank y es uno (no el \'unico) de los criterios utilizados para ponderar la importancia
de los resultados de una b\'usqueda. En este trabajo nos concentraremos en el estudio y desarrollo del algoritmo
PageRank.

\textbf{El problema}

El algoritmo PageRank se basa en la construcci\'on del siguiente modelo. Supongamos que tenemos una red con $n$ p\'aginas 
web $Web = \{1,\dots,n\}$ donde
el objetivo es asignar a cada una de ellas un puntaje que determine la importancia relativa de la misma respecto de las
dem\'as. Para modelar las relaciones entre ellas, definimos la \emph{matriz de conectividad} $W \in \{0,1\}^{n \times n}$ 
de forma tal que $w_{ij} = 1$ si la p\'agina $j$ tiene un link a la p\'agina $i$, y $w_{ij} = 0$ en caso contrario. 
Adem\'as, ignoramos los \emph{autolinks}, es decir, links de una p\'agina a s\'i misma, definiendo $w_{ii} = 0$. Tomando 
esta matriz, definimos el grado de la p\'agina $j$, $n_j$, como la cantidad de links salientes hacia otras p\'aginas 
de la red, donde $n_j = \sum_{i = 1}^n w_{ij}$. Adem\'as, notamos con $x_j$ al puntaje asignado a la p\'agina $j\in
Web$, que es lo que buscamos calcular.

La importancia de una p\'agina puede ser modelada de diferentes formas. Un link de la p\'agina $u \in
Web$ a la p\'agina $v \in Web$ puede ser visto como que $v$ es una p\'agina importante. Sin embargo, no queremos que una
p\'agina obtenga mayor importancia simplemente porque es apuntada desde muchas p\'aginas. 
Una forma de limitar esto es ponderar los links utilizando la importancia de la p\'agina de origen. En otras palabras,
pocos links de p\'aginas importantes pueden valer m\'as que muchos links de p\'aginas poco importantes. En particular,
consideramos que la importancia de la p\'agina $v$ obtenida mediante el link de la p\'agina $u$ es proporcional a la 
importancia de la p\'agina $u$ e inversamente proporcional al grado de $u$. Si la p\'agina $u$ contiene $n_u$ links,
uno de los cuales apunta a la p\'agina $v$, entonces el aporte de ese link a la p\'agina $v$ ser\'a $x_u/n_u$. Luego,
sea $L_k \subseteq Web$ el conjunto de p\'aginas que tienen un link a la p\'agina $k$. Para cada p\'agina pedimos que
\begin{eqnarray}
x_k = \sum_{j \in L_k} \frac{x_j}{n_j},~~~~k = 1,\dots,n. \label{eq:basicmodel}
\end{eqnarray}
Definimos $P \in  \mathbb{R}^{n \times n}$ tal que $p_{ij} = 1/n_j$ si $w_{ij} = 1$, y $p_{ij} = 0$ en caso contrario. Luego,
el modelo planteado en (\ref{eq:basicmodel}) es equivalente a encontrar un $x\in \mathbb{R}^n$ tal que $Px = x$, es
decir, encontrar (suponiendo que existe) un autovector asociado al autovalor 1 de una matriz cuadrada, tal que $x_i \ge
0$ y $\sum_{i = 1}^n x_i = 1$. En
Bryan y Leise \cite{Bryan2006} y Kamvar et al. \cite[Secci\'on 1]{Kamvar2003} se analizan ciertas condiciones que debe
cumplir la red de p\'aginas para garantizar la existencia de este autovector.

Una interpretaci\'on equivalente para el problema es considerar al \emph{navegante aleatorio}. \'Este empieza en una
p\'agina cualquiera del conjunto, y luego en cada p\'agina $j$ que visita sigue navegando a trav\'es de sus links,
eligiendo el mismo con probabilidad $1/n_j$. Una situaci\'on particular se da cuando la p\'agina no tiene links salientes. En
ese caso, consideramos que el navegante aleatorio pasa a cualquiera de las p\'agina de la red con probabilidad $1/n$.
Para representar esta situaci\'on, definimos $v \in \mathbb{R}^{n \times n}$, con $v_i = 1/n$ y $d \in \{0,1\}^{n}$ donde 
$d_i = 1$ si $n_i = 0$, y $d_i = 0$ en caso contrario. La nueva matriz de transici\'on es 
\begin{eqnarray*}
D & = & v d^t \\
P_1 & = & P + D.
\end{eqnarray*}
Adem\'as, consideraremos el caso de que el navegante aleatorio, dado que se encuentra en la p\'agina $j$, decida visitar
una p\'agina cualquiera del conjunto, independientemente de si esta se encuentra o no referenciada por $j$ (fen\'omeno
conocido como \emph{teletransportaci\'on}). Para ello, consideramos que esta decisi\'on se toma con una probabilidad
$c \ge 0$, y podemos incluirlo al modelo de la siguiente forma:
\begin{eqnarray*}
E & = & v \bar{1}^t \\
P_2 & = & cP_1 + (1-c)E,
\end{eqnarray*}
\noindent donde $\bar{1} \in \mathbb{R}^n$ es un vector tal que todas sus componenetes valen 1. La matriz resultante
$P_2$ corresponde a un enriquecimiento del modelo formulado en (\ref{eq:basicmodel}). Probabil\'isticamente, la
componente $x_j$ del vector soluci\'on (normalizado) del sistema $P_2 x = x$ representa la proporci\'on del tiempo que,
en el largo plazo, el navegante aleatorio pasa en la p\'agina $j \in Web$.

En particular, $P_2$ corresponde a una
matriz \emph{estoc\'astica por columnas} que cumple las hip\'otesis planteadas en Bryan y Leise \cite{Bryan2006} y
Kamvar et al. \cite{Kamvar2003}, tal que $P_2$ tiene un autovector asociado al autovalor 1, los dem\'as autovalores de
la matriz cumplen $1 = \lambda_1 > |\lambda_2| \ge \dots \ge |\lambda_n|$ y, adem\'as, la dimensi\'on
del autoespacio asociado al autovalor $\lambda_1$ es 1. Luego, la soluci\'on al sistema $P_2 x = x$ puede ser calculada
de forma est\'andar utilizando el m\'etodo de la potencia.

\textbf{Enunciado}

El objetivo del trabajo es implementar el algoritmo PageRank, considerando dos m\'etodos distintos para el c\'alculo del
autovector principal de la matriz $P_2$. Para ello, se considera el entorno de aplicaci\'on real del algoritmo, donde el
n\'umero total de p\'aginas, $n$, es considerablemente grande (i.e., todas las p\'aginas web accesibles p\'ublicamente).
El programa tomar\'a como entrada un archivo con la representaci\'on de grafo de conectividad, construir\'a la matriz
$P_2$ definida en la secci\'on anterior y ejecutar\'a el algoritmo PageRank utilizando el m\'etodo de la potencia y una 
variante del mismo para distintas instancias de prueba. Se pide:

\begin{enumerate}
\item En base a su definici\'on, $P_2$ no es una matriz esparsa. Sin embargo, en Kamvar et al. \cite[Algoritmo
1]{Kamvar2003} se propone una forma alternativa para computar $x^{(k+1)} = P_2 x^{(k)}$. Mostrar que el c\'omputo
propuesto es equivalente. Utilizarlo para mejorar el espacio requerido en memoria para el almacenamiento de la matriz
$P_2$ y el tiempo de ejecuci\'on requerido para hacer la multiplicaci\'on entre matrices y vectores. 

\item Bas\'andose en el an\'alisis del punto anterior, implementar el m\'etodo de la potencia para calcular el
autovector principal de la matriz $P_2$.

\item Implementar la variante del M\'etodo de la Potencia propuesta en Kamvar et al. \cite[Secci\'on 5]{Kamvar2003},
denominada Extrapolaci\'on Cuadr\'atica. El m\'etodo de Cuadrados M\'inimos involucrado debe ser resuelto utilizando la
Factorizaci\'on QR de la matriz mediante alguno de los m\'etodos vistos en la materia.

\item Realizar experimentos considerando distintas instancias de prueba. Para ello, se podr\'a utilizar el c\'odigo
adjuntada para la generaci\'on de instancias en base a datos reales, o cualquier otra herramienta que el grupo considere
necesaria. Evaluar tambi\'en los algoritmos alguno de los conjuntos de instancias
provistos en \cite{SNAP}. Para cada algoritmo, analizar el tiempo de
ejecuci\'on, la evoluci\'on del error entre iteraciones consecutivas y considerar distintos criterios de parada. 
Adem\'as, analizar la calidad del ordenamiento obtenido en t\'erminos de la relevancia de las p\'aginas.
\end{enumerate}

\textbf{Formato de archivos}

El programa deber\'a recibir como par\'ametro un archivo con la representaci\'on esparsa de la matriz (grafo) de
conectividad. El archivo contendr\'a una l\'inea con la cantidad de p\'aginas ($n$), la cantidad de links ($m$) y luego
una lista con un link por l\'inea, indicando la p\'agina de origen y destino separadas por un espacio. A modo de
ejemplo, a continuaci\'on se muestra el archivo de entrada correspondiente al ejemplo propuesto en Bryan y Leise
\cite[Figura 1]{Bryan2006}: 

\begin{verbatim}
4 
8 
1   2
1   3
1   4
2   3
2   4
3   1
4   1
4   3
\end{verbatim}

Una vez ejecutado el algoritmo, el programa deber\'a generar un archivo de salida que contenga una l\'inea por cada
p\'agina, acompa\~nada del puntaje obtenido por el algoritmo PageRank, ordenados en forma decreciente en funci\'on de
este \'ultimo valor.

Para generar instancias, es posible utilizar el c\'odigo Python provisto por la c\'atedra. Es importante mencionar que, para que el mismo funcione, es
necesario tener acceso a Internet. En caso de encontrar un bug en el mismo, por favor contactar a los docentes de la
materia a trav\'es de la lista. Desde ya, el c\'odigo puede ser modificado por los respectivos grupos agregando todas
aquellas funcionalidades que consideren necesarias.

\vskip 15pt

\hrule

\vskip 11pt


{\bf \underline{Fechas de entrega}}
\begin{itemize}
 \item \emph{Formato Electr\'onico:} Domingo 10 de Noviembre de 2013, hasta las 23:59 hs, enviando el trabajo (informe +
 c\'odigo) a la direcci\'on \verb+metnum.lab@gmail.com+. El subject del email debe comenzar con el texto \verb+[TP3]+
 seguido de la lista de apellidos  de los integrantes del grupo.
 \item \emph{Formato f\'isico:} Lunes 11 de Noviembre de 2013, de 17 a 18 hs.
\end{itemize}

\noindent \textbf{Importante:} El horario es estricto. Los correos recibidos despu\'es de la hora indicada ser\'an considerados re-entrega.  

\bibliographystyle{plain}
\bibliography{tp3}

\end{document}

