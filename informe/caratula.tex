%\rule{\linewidth}{1 cm}
\rule{16 cm}{0.5 mm}

\vspace{2cm}

Integrantes:
\begin{itemize}
	\item Castro, Dami\'an L.U.: 326/11  \verb+ltdicai@gmail.com+
	\item Matayoshi, Leandro L.U.: 79/11 \verb+leandro.matayoshi@gmail.com+
	\item Szyrej, Alexander L.U.: 642/11   \verb+alexanderszyrej@gmail.com+
	
\end{itemize}

\vspace{2cm}

\begin{abstract}
	En 1996, estudiantes de Stanford comenzaron a desarrollar un nuevo motor de b\'usqueda con un principio en mente: Popularidad de p\'aginas web. Para lograr esto llevaron a cabo un sistema llamado \emph{PageRank}, un m\'etodo que les otorga puntaje a cada una de las p\'aginas en toda la red utilizando un an\'alisis probabil\'istico. El siguiente informe analizar\'a algunas implementaciones para resolver dicho sistema y sus fundamentos de por qu\'e son correctas. Primero veremos el m\'etodo de la potencia con una optimizaci\'on que aprovecha el hecho que la matriz proviene de una matriz esparsa y luego analizaremos el m\'etodo de extrapolaci\'on cuadr\'atica para acelerar el m\'etodo anterior. Como \'ultimo veremos como se comparan ambos m\'etodos y discutiremos los resultados.
\end{abstract}

\vspace{2cm}

Palabras Clave:
\begin{itemize}
	\item PageRank
	\item Extrapolaci\'on Cuadr\'atica
	\item Matriz esparsa
\end{itemize}

