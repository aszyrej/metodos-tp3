\subsection{Enunciado}

El objetivo del trabajo es implementar el algoritmo PageRank, considerando dos m\'etodos distintos para el c\'alculo del
autovector principal de la matriz $P_2$. Para ello, se considera el entorno de aplicaci\'on real del algoritmo, donde el
n\'umero total de p\'aginas, $n$, es considerablemente grande (i.e., todas las p\'aginas web accesibles p\'ublicamente).
El programa tomar\'a como entrada un archivo con la representaci\'on de grafo de conectividad, construir\'a la matriz
$P_2$ definida en la secci\'on anterior y ejecutar\'a el algoritmo PageRank utilizando el m\'etodo de la potencia y una 
variante del mismo para distintas instancias de prueba. Se pide:

\begin{enumerate}
\item En base a su definici\'on, $P_2$ no es una matriz esparsa. Sin embargo, en Kamvar et al. \cite[Algoritmo
1]{Kamvar2003} se propone una forma alternativa para computar $x^{(k+1)} = P_2 x^{(k)}$. Mostrar que el c\'omputo
propuesto es equivalente. Utilizarlo para mejorar el espacio requerido en memoria para el almacenamiento de la matriz
$P_2$ y el tiempo de ejecuci\'on requerido para hacer la multiplicaci\'on entre matrices y vectores. 

\item Bas\'andose en el an\'alisis del punto anterior, implementar el m\'etodo de la potencia para calcular el
autovector principal de la matriz $P_2$.

\item Implementar la variante del M\'etodo de la Potencia propuesta en Kamvar et al. \cite[Secci\'on 5]{Kamvar2003},
denominada Extrapolaci\'on Cuadr\'atica. El m\'etodo de Cuadrados M\'inimos involucrado debe ser resuelto utilizando la
Factorizaci\'on QR de la matriz mediante alguno de los m\'etodos vistos en la materia.

\item Realizar experimentos considerando distintas instancias de prueba. Para ello, se podr\'a utilizar el c\'odigo
adjuntada para la generaci\'on de instancias en base a datos reales, o cualquier otra herramienta que el grupo considere
necesaria. Evaluar tambi\'en los algoritmos alguno de los conjuntos de instancias
provistos en \cite{SNAP}. Para cada algoritmo, analizar el tiempo de
ejecuci\'on, la evoluci\'on del error entre iteraciones consecutivas y considerar distintos criterios de parada. 
Adem\'as, analizar la calidad del ordenamiento obtenido en t\'erminos de la relevancia de las p\'aginas.
\end{enumerate}

\subsection{Correctitud del Algoritmo 1 de Kamvar} 
Seg\'un Kamvar et al. se puede reducir el problema de $(P_{2})x = y$ a realizar el siguiente c\'alculo:

	$a = cPx$

	$b = ||x||_{1} - ||a||_{1}$

	$y = a + bv$

Veamos que eso es correcto desarrollando $(P_{2})x$.

\begin{align}
	& P_{2}x = [cP_{1} + (1-c)E]x = [c(P + D) + (1-c)E]x = cPx + cDx + Ex -cEx \nonumber\\
	& \implies cPx + cDx + Ex -cEx = y= cPx +bv \label{ap1}
\end{align}

Se puede observar en \ref{ap1} que ya logramos que se cumpla el primer t\'ermino. Veamos ahora que  

\begin{align*}
	cDx + Ex -cEx &= bv \\
	cvd^{t}x + v[1]^{t}x - cv[1]^{t}x &= bv
\end{align*}

Como $d^{t}x$ y $[1]^{t}x$ son escalares, puedo moverlos a la izquierda del producto

\begin{align}
	(cd^{t}x)v + ([1]^{t}x)v - (c[1]^{t}x)v &= bv \nonumber\\
	(cd^{t}x +[1]^{t}x - c[1]^{t}x)v &= bv \label{ap2}
\end{align}

Entonces, por \eqref{ap2} quiero ver que:
\begin{align}
	(cd^{t}x +[1]^{t}x - c[1]^{t}x) = b = ||x||_{1} - ||a||_{1}\label{ap3}
\end{align}
Se puede observar que como x es un vector de probabilidades entonces todas sus coordenadas son no negativas, y por lo tanto $[1]^{t}x = ||x||_{1}.$

Reemplazo esto en la ecuaci\'on \eqref{ap3}

\begin{align}
(cd^{t}x +[1]^{t}x - c[1]^{t}x) &= [1]^{t}x - ||cPx||_{1} \nonumber\\
(cd^{t}x-c[1]^{t}x) &= -||cPx||_{1} \nonumber\\
-c([1]^{t}x-d^{t}x) &= -c||Px||_{1} \nonumber\\
([1]^{t} -d^{t})x &= ||Px||_{1} = \left(\sum_{i=i}^{n}[fila_{i}(P)]x\right) = \left(\sum_{i=i}^{n}(fila_{i}(P))\right)x \label{ap4}
\end{align}

Entonces bastar\'ia con ver que, $\forall$ $ 1 \leq j \leq n$:

\begin{align}
	([1]^{t} -d^{t})_{j} = \left(\sum_{i=i}^{n}(fila_{i}(P))\right)_{j} \label{ap5}
\end{align}

Analicemos \eqref{ap5} por separado.

Primer lado de la ecuaci\'on. Siendo $d_{out}(k)$ el n\'umero de vecinos de la p\'agina $k$.
\[
	([1]^{t} -d^{t})_{j} = 
	1 - d_{j} = 
	\begin{cases}
		1 & \text{si }d_{j} = 0 \iff d_{out}(j) \neq 0 \\
		0 & \text{si no}
	\end{cases}
\]

Segundo lado de la ecuaci\'on. Como la matriz $P$ tiene todos sus elementos no negativos entonces vale:

\[
	\left(\sum_{i=i}^{n}(fila_{i}(P))\right)_{j} = ||columna_{j}(P)||_{1} =
		\begin{cases}
		1 & \text{si la p\'agina $i$ tiene links salientes } \iff d_{out}(j) \neq 0 \\
		0 & \text{si no}
	\end{cases}
\]

Esto se debe a que la matriz $P$, cuando la columna tiene elementos distintos de 0 entonces cumple que la suma de los elementos de la columna es igual a 1.

Por lo tanto, para mismos valores de $j$ ambos lados de la ecuaci\'on valen lo mismo, entonces puedo asegurar que son iguales. Y con esto demuestro que utilizar el Algoritmo 1 de Kamvar\cite{Kamvar2003} es equivalente a resolver $P_{2}x = y$