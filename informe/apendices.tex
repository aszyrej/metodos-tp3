\subsection{Correctitud del Algoritmo 1 de Kamvar} 
Seg\'un Kamvar et al. se puede reducir el problema de $(P_{2})x = y$ a realizar el siguiente c\'alculo:

	$a = cPx$

	$b = ||x||_{1} - ||a||_{1}$

	$y = a + bv$

Veamos que eso es correcto desarrollando $(P_{2})x$.

\begin{align}
	& P_{2}x = [cP_{1} + (1-c)E]x = [c(P + D) + (1-c)E]x = cPx + cDx + Ex -cEx \nonumber \\
	& \implies cPx + cDx + Ex -cEx = y= cPx +bv 
\end{align}

Se puede observar en (1) que ya logramos que se cumpla el primer t\'ermino. Veamos ahora que  

\begin{align*}
	cDx + Ex -cEx &= bv \\
	cvd^{t}x + v[1]^{t}x - cv[1]^{t}x &= bv
\end{align*}

Como $d^{t}x$ y $[1]^{t}x$ son escalares, puedo moverlos a la izquierda del producto

\begin{align}
	(cd^{t}x)v + ([1]^{t}x)v - (c[1]^{t}x)v &= bv \nonumber\\
	(cd^{t}x +[1]^{t}x - c[1]^{t}x)v &= bv
\end{align}

Entonces, por (2) quiero ver que:
\begin{align}
	(cd^{t}x +[1]^{t}x - c[1]^{t}x) = b = ||x||_{1} - ||a||_{1}
\end{align}
Se puede observar que como x es un vector de probabilidades entonces todas sus coordenadas son no negativas, y por lo tanto $[1]^{t}x = ||x||_{1}.$

Reemplazo esto en la ecuaci\'on (3)

\begin{align}
(cd^{t}x +[1]^{t}x - c[1]^{t}x) &= [1]^{t}x - ||cPx||_{1} \nonumber\\
(cd^{t}x-c[1]^{t}x) &= -||cPx||_{1} \nonumber\\
-c([1]^{t}x-d^{t}x) &= -c||Px||_{1} \nonumber\\
([1]^{t} -d^{t})x &= ||Px||_{1} = \left(\sum_{i=i}^{n}[fila_{i}(P)]x\right) = \left(\sum_{i=i}^{n}(fila_{i}(P))\right)x
\end{align}

Entonces bastar\'ia con ver que, $\forall$ $ 1 \leq j \leq n$:

\begin{align}
	([1]^{t} -d^{t})_{j} = \left(\sum_{i=i}^{n}(fila_{i}(P))\right)_{j}
\end{align}

Analicemos (5) por separado.

Primer lado de la ecuaci\'on. Siendo $d_{out}(k)$ el n\'umero de vecinos de la p\'agina $k$.
\[
	([1]^{t} -d^{t})_{j} = 
	1 - d_{j} = 
	\begin{cases}
		1 & \text{si }d_{j} = 0 \iff d_{out}(j) \neq 0 \\
		0 & \text{si no}
	\end{cases}
\]

Segundo lado de la ecuaci\'on. Como la matriz $P$ tiene todos sus elementos no negativos entonces vale:

\[
	\left(\sum_{i=i}^{n}(fila_{i}(P))\right)_{j} = ||columna_{j}(P)||_{1} =
		\begin{cases}
		1 & \text{si la p\'agina $i$ tiene links salientes } \iff d_{out}(j) \neq 0 \\
		0 & \text{si no}
	\end{cases}
\]

Esto se debe a que la matriz $P$, cuando la columna tiene elementos distintos de 0 entonces cumple que la suma de los elementos de la columna es igual a 1.

Por lo tanto, para mismos valores de $j$ ambos lados de la ecuaci\'on valen lo mismo, entonces puedo asegurar que son iguales. Y con esto demuestro que utilizar el Algoritmo 1 de Kamvar es equivalente a resolver $P_{2}x = y$

